\documentclass[11pt]{article}
\usepackage[utf8]{inputenc}
\usepackage{graphicx}
\usepackage[italian]{babel}
\usepackage{hyperref}
\usepackage{float}
\title{Progetto di Tecnologie Web}


\begin{document}
	\maketitle
	\begin{figure}[h]
		\centering
		\includegraphics[width=0.7\linewidth]{logo-unipd.png}
	\end{figure}
	\begin{center}{\fontsize{20}{10}\selectfont Autori:}\end{center}
	\begin{center}{\fontsize{20}{30}\selectfont 
			Marco Casagrande
			\\ Walter Sandon 1009138
			
			}\end{center}
	
	\newpage
	\tableofcontents
	\newpage
	\listoffigures
	\newpage
	
	\section{Abstract}
	
Il progetto sviluppato implementa il sito internet di una azienda che produce per conto terzi griglie metalliche ad uso domestico, quali griglie per forni e frigoriferi.
Il sito si propone di dare ogni informazione che possa essere rilevante agli utenti visitatori come i prodotti che la IMAS riesce a realizzare e le relative lavorazioni disponibili, la posizione della ditta  e la possibilità, nel caso si abbia la necessità di ulteriori delucidazioni, di potere contattere gli opratori tramite apposito form nell'area dedicata.
Inoltre gli amministratori hanno la possibilità di accedere ad un area riservata dalla quale modificare  o aggiungere i prodotti e lavorazioni in catalogo; in particolare lato prodotti:
\begin{itemize}
	\item inserire nuovo prodotto,
	\item assegnare a più prodotti una lavorazione, 
	\item modificare un prodotto esistente, in particolare:	
	\begin{itemize}
		\item nome
		\item foto
		\item attributo alt della foto
		\item lavorazione
		\item descrizione
	\end{itemize}
	\item eliminare un prodotto.
\end{itemize}
La lavorazione è similare al prodotto in quanto di possono eseguire le stesse operazioni di inserimento e modifica dei propri campi

\subsection{Combinazione dei colori}
Per cercare di garantire una totale usabilità del sito anche ad utenti affetti da problemi visivi è stato scelto di utilizzare uno schema a colori che esaltasse il contrasto tra sfondo e testo. Per testare tale scelta è stato utilizzato il servizio offerto da \href{http://colorfilter.wickline.org/}{\textit{wickline}} che mostra come il sito può venire visualizzato da utenti con determinati problemi. Di seguito vengono riportati i risultati ottenuti.
\newpage
\section{Utenti destinatari}
\paragraph{Privati/Aziende}
Una prima categoria di utenti destinatari del sito comprende in piccola misura i privati cittadini e nella maggioranza alle aziende che producono oggetti che rientrano nelle categorie merciologiche presenti nella sezione  \textit{Prodotti}.\\
Questi utenti potranno visualizzare il catalogo e constatare la presenza di un particolare prodotto; per questo, oltre alle informazioni di base del prodotto, si può trovare l'indicazione circa le possibili \textit{lavorazioni}.\\
Oltre a ciò, gli utenti potranno (nelle due pagine \textit{Home} e \textit{Contatti}) scoprire la storia della compagnia e qual'ora fossero interessati, chiedere informazioni tramite un apposito form.

\paragraph{Amministratori}
Un'altra categoria di utenti del sito è rappresentata dagli amministratori IMAS che potranno accedere, autenticandosi attraverso un link nell' \textit{header} a destra del logo aziendale, ad un'area riservata dalla quale è possibile gestire il catalogo dei prodotti e lavorazioni che può offrire la ditta.

\newpage

\section{Usabilità}

Dopo aver discusso col proponente riguardo alle possibili implementazioni del sito, in fase di progettazione ci siamo chiesti come distribuire le varie informazioni nel sito cercando di dare una giusta collocazione alle cose ponendo l'attenzione su due variabli quali le azioni che un utente maggiormente esegue e la tipologia aziendale in questione

\begin{description}
	\item [Consultare il catalogo prodotti] \hfill \\
	L'operazione più frequentemente compiuta dagli utenti è quella di consultare il catalogo dei prodotti presenti.
	\\
	Essendo però una ditta che non ha una continuità di rinnovo dei prodotti, si è ritenuto apportuno non metterlo in evidenza nella \textit{homepage}, ma piuttosto  dedicare un area apposita. 
	La \textbf{Home} è stata adibita a chi entra per la prima volta nel sito o piuttosto qualche vecchio utente che vogliano informarsi circa sulle origini, sulla storia o sulla mission aziendale. \\
	Nel contempo in queste sezioni sono stati inseriti link diretti per poter consultare i nostri prodotti  ed è stata inoltre predisposta una comoda \textit{NavBar} con la quale raggiungere comodamente le pagine \textbf{Prodotti} e \textbf{Lavorazioni}, i cui nomi sono esplicativi circa le informazioni contenute.
	Stessa considerazione va fatta per la \textbf{consultazione delle lavorazioni}.

	\item [Manutenzione del catalogo] \hfill \\
	L'operazione di manutenzione dei cataloghi (aggiunta, modifica e rimozione dei prodotti/lavorazioni) è riservata agli amministratori del sito.\\
	 È logico pensare che, essendo tali utenti un numero inferiore rispetto ai visitatori, il link per l'autenticazione e l'accesso all'area riservata possa essere messa in posizione \textit{nascosta}. Portroppo per ragioni di spazio l'abbiamo posizionata solo in alto a destra sapendo di creare così disorientamento.\\
	 Abbiamo cercato di limitare i danni avvisando,  una volta entrati nell'area login, che tale area è riservata ai dependenti IMAS.\\
	  \textit{Si noti che il pulsante per poter entrare nell'area gestione prodotti e lavorazioni è visibile solo quando  l'amministratore ha eseguito il login.}
\end{description}
\subsection{Elementi dell'Interfaccia Grafica}

Analizziamo ora gli elementi dell'interfaccia utilizzati per renderla il più chiara (intuitiva per ogni utente) e diretta (minor numero di click per un operazione) possibile.
\begin{description}
	\item [NavBar] La barra di navigazione è presente in ogni pagina e consente l'accesso diretto alla homepage, l'accesso alle pagine \textit{Prodotti}, \textit{Lavorazioni} e \textit{Contatti}.
	\\È stata progettata in modo tale che l'utente sappia in ogni momento in che pagina si trova. Infatti la parte della barra corrispondente alla pagina visitata risulta essere evidenziata.
	\item [Breadcrumbs] Per evitare che l'utente si senta disorientato all'interno del sito, quando dalla homepage si addentra con maggiore profondità nella struttura del sito, compare il percorso dell'utente a partire dalla homepage.
	\item [Link] Abbiamo deciso di adottare due tipologie di link, quelli che si trovano immersi nel testo e quelli che invece sono inseriti nelle due sezioni precedentemente descritte.
	\\ I primi sono sottolineati e aderiscono alla rappresentazione standard dei link ovvero blu se non visitati e viola in caso contrario.Tutti gli altri inseriti nelle sezioni \textit{nav bar} e \textit{Breadcrumbs} sono comunque ben visibili con il passaggio del mouse grazie all'effetto \textbf{hover}.
\end{description}

\newpage
\section{Gerarchia file}

La gerarchia del sito è formata da 3 cartelle:
\begin{itemize}
	\item \texttt{cgi-bin} con gli script \texttt{.cgi}
	\item \texttt{data} con 3 sotto-cartelle
	\begin{itemize}
		\item \texttt{xml} contente tutti i file xml
		\item \texttt{xsd} contente gli \textit{XMLSchema}
		\item \texttt{xsl} contenete i fogli di trasformazione
	\end{itemize}
	\item \texttt{public\_html} con il file \texttt{index.xhtml} e le sotto-cartelle:
	\begin{itemize}
		\item \texttt{images} con le immagini che ci sono servite per decorare il sito
		\item \texttt{icons} con l'e icone per i tab dei browser
		\item \texttt{javascript} con le funzioni appunto in \textit{JavaScript}
		\item \texttt{parts} con i file xhtml parziali, ovvero parti di codice xhtml che verranno stampate dinamicamente grazie al perl
		\item \texttt{css} con i fogli di stile  
	\end{itemize}
\end{itemize}
\newpage
\section{Architettura}
\subsection{Progettazione layout}
Prima di iniziare la stesura del codice abbiamo cercato di organizzare i dati che volevamo rappresentare. Questo ci ha permesso di chiarire alcuni aspetti fondamentali che il sito doveva avere. Innanzitutto la semplicità di navigazione al suo interno. Se un sito è semplice è visitabile e capibile da più persone. Il target d'utenza è l'utente in grado di utilizzare strumenti per la navigazione sul web.
\\
\\
Fin da subito abbiamo optato per un layout di tipo fluido. Ha richiesto più lavoro del previsto ma ha ricompensato lo sforzo in quanto, abbiamo dovuto adattarlo ai vari tipi di monitor cambiando di molto l'aspetto, ma questo ha reso possibile un alto indice di accessibilità.
\subsection{Sviluppo layout}
Nell'immagine sottostante è possibile visualizzare la struttura principale dei div, i quali raccolgono per contenuto tematico le informazioni.
Il layout adottato ha come larghezza massima 1200px dovuto ad un fattore estetico dato che le varie sezioni del \textit{Container} cambia di colore ad ogni cambio d'informazione, mentre l'altezza non è stata in nessun modo alterata rispetto l'originale. Questo permette una evoluzione verso il basso senza problemi.
\begin{figure}[H]	
	\centering
	\includegraphics[width=\linewidth]{layout.png}
	\caption{Layout delle pagine}
	\label{Layout delle pagine}
\end{figure}

Per quanto riguarda il nostro contesto, la maggioranza degli utenti visitatori saranno aziende interessate alla nostra ditta pertanto si è deciso di puntare ad un layout adatto a partatili piuttosto che a schermi di grandi dimensioni. Non abbiamo sviluppato la parte mobile lato smartphone garantendo comunque che i contenuti non subiranno alterazioni spiacevoli fino a quando la grandezza dello schermo sarà grande come i table in modalità portrait. 
\\
\\
Tutto il layout è stato progettato a pannelli adattabili, in questo modo si ha una più espandibilità.
\begin{itemize}
	\item Partendo con l'analisi dall'alto si evidenzia subito il div "header", esso ha il compito di informare l'utente su ciò che sta visitando.
	\\In questo caso siamo stati un pò condizionati dal fatto che non avevamo una totale libertà. Infatti manca una breve descrizione per far capire all'utent quale sia la tematica del sito.
	\\Come spiegato in precedenza a lato a destra è presente il link per l'accesso agli amministratori.
	\item Sempre nell'area di maggior visibilita è presente il menu utente. Anch'esso è stato identificato tramite l'utilizzo del div "menu". Presenta 4 campi rappresentati da \textbf{Home}, \textbf{Prodotti}, \textbf{Lavorazioni} e \textbf{Contatti}. Evidenzia il menù attivo in ogni pagina del sito. Questa area rappresenta l'asse informativo di dove un utente può andare.
	\item Sottostante il menu principale è presente il div "breadcrumbs". Esso ha il compito di aiutare l'utente ad identificare in che posizione del sito si trova rispetto l’homepage del sito. Questa area rappresenta l'asse informativo di dove l'utente sia in quel determinato istante.
	\item Il blocco centrale è rappresentato dal div "container". Ha il compito di raccogliere la sezione principale, tutte le informazioni raccolte dagli utenti verranno rappresentate all'intero di esso. Questa area rappresenta l'asse informativo di cosa si sta trattando.
	\item Nella parte inferiore del sito è stato creato il div "footer", al suo interno sono presenti le credenziali della ditta e i loghi della certificazione html e css.
\end{itemize}

\subsection{Layout per dispositivi mobili}
Il layout per i dispositivi mobili è stato sviluppato per risoluzioni non inferiori ai 768px, l'aspetto del sito non varia, ma si è cercato di ridimensionare le dimensioni dei font piuttosto che la grandezza delle immagine in modo da utilizzare quanta più area possibile per la visualizzazione.
\\Ove non è stato possibile abbiamo cambiato la disposizione degli oggetti.

\subsection{Layout di stampa}
Nel layout di stampa si è deciso di eliminare tutte le informazioni superficiali e di mantenere il contenuto semplice e privo di colori e immagini. Vista la natura commerciale del sito, si vuole offrire una stampa che riporti solo le informazioni chiave all'utente. In particolare, saranno le sezioni Prodotti e Lavorazioni ad essere oggetti di stampa, che infatti riportano le immagini dei vari prodotti ed una lista di rapida consultazione. 
\\Le sezioni amministrative mantengono lo stesso stile minimale, considerata anche la scarsa probabilità di stamparle. Nel layout di stampa i font non sono stati modificati rispetto a quelli di default, in modo da non alterare la relativa qualità di stampa o portabilità nel caso di stampa in PDF.
\newpage
\section{Struttura}
La struttura del sito è stata suddivisa per contenuti, in modo da aiutare l'utente nella ricerca delle informazioni. Questo porta ad una maggiore chiarezza delle sezioni presenti.
\\
\\
Tutto il sito è stato sviluppato in XHTML 1.0 Strict. Le pagine sono generate da PERL tramite sistemi di templating. L'unica eccezione riguarda la pagina Contatti, che necessitava di alcuni elementi di HTML5.
\\
\\Di seguito vengono spiegate le pagine principali:
\\
\begin{description}
	\item[public\_html/index.html] \hfill \\
	 tramite l'utilizzo della funzione \textit{http-equiv="refresh"} abbiamo reso possibile il reindirizzamento alla homepage che si trova all’interno della cartella "cgi-bin"nel caso l'untente acceda alla directory principale del sito;
	 \item[cgi-bin/index.cgi]  la pagina viene generata con un contenuto esclusivamente statico richiamando altre parti comuni e specifiche della pagina che si trovano in \textbf{public\_html/parts}. La parte comune in ogni pagina risulta essere solo il footer, metre l'header e il menu, se pur considerati statici, sono specifici alla pagina;
	 \item[cgi-bin/prodotti.cgi]  questa pagina viene generata da un contenuto sia dinamico che statico. Dove la parte statica è formata dall'header, menù e footer, mentre il contenuto presente nel \textit{container} è generata dinamicamente richiamando "../data/xsl/prodotti.xslt" che estrae dell'xml i dati necessari per la composizione della pagina;
	 \item[cgi-bin/lavorazioni.cgi]  come per la pagina precedente anche la seguente è composta sia da contenuto statico che dinamico. La parte dinamica si differenzia dal fatto che non viene richiamato lo stesso file \textit{.xslt} ma quello specidico alle lavorazioni. 
	 \item[cgi-bin/contatti.cgi] questa pagina è costruita come l' \textit{homepage} in quanto il suo contenuto è totalmente statico. La particolarità di questa pagina è che richiama del codice javascript contenuto in \textit{public\_html/javascript} per inviare una richiesta d'informazione alla ditta dopo aver premuto il tasto \textbf{invia} dell'apposita form presente nella pagina.
	 \\La pagina offre degli aiuti nel caso si inseriscano dei dati errati;
	 \item[cgi-bin/login.cg] la seguente pagina permette l'autenticazione degli amministratori per poter accedere all'area riservata. Anche in questo caso pagina offre degli aiuti nel caso si inseriscano dei dati errati
	 \\Dal momento in cui ci si logga in ogni pagina al posto del link \textit{area riservata} saranno presenti 2 link per l'area \textbf{amministrazione}e per il \textbf{loguot};
	 \item[cgi-bin/admin.cg] il ruolo fondamentale della pagina è l'inserimento dei dati. Si presenta con 2 sezioni: \textit{Prodotti} e  \textit{Lavorazioni} e ognuna delle quali presenta le operazioni che l'amministratore può fare ad ognu.	 
\end{description}
\newpage

\section{Presentazione}
Per la presentazione delle informazioni abbiamo cercato di essere quanto più precisi possibili ed accessibili. Abbiamo deciso di sviluppare l'interfaccia grafica tramite l'utilizzo di codice CSS. Volendo ottenere un ottimo grado di accessibilità abbiamo deciso di utilizzare CSS in versione 2, in quanto compatibile con la maggioranza dei browser in circolazione.
Riteniamo che anche se la pagina dovesse essere visualizzata da browser con compatibilità inferiore il contenuto e l'accessibilità non vengono alterati.
\\
Tutte le pagine mostrano una grafica diversa, simili solo nell'alternanza del colore al cambio informazione. Questo è statto necessario in quanto ogni pagina presenta un'ospetto deverso non compatibile semanticamente con quello delle altre pagine.
\\ Siamo riusciti a presentare un design accativante e piacevole alla vista, di contro invece,  proprio per l'incompatibilità appena descritta, non siamo riusciti ad usare in più pagine gli stessi elementi portandoci così ad avere un CSS leggermente prolisso e suscettibile di errori. 
\\Nonostante ciò crediamo che la visualizzazione sia ottimale in ugual modo. 
\\
\\
Ai fini del miglioramento della presentazione e delle performance abbiamo deciso di suddividere i file CSS in base alle funzioni a loro riservate. Esistono 3 tipi di CSS:
\begin{itemize}
	\item Il file "default.css" utilizzato per la maggior parte dei dispositivi con risoluzione maggiore.
	\item Il file "mobile.css" utilizzato per la parte dei dispostivi mobili quali i tablet, dove la dimensione dello schermo rimane più compatta. Esso viene attivato ad una risoluzione inferiore ai 768px;
	\item Il file "print.css" utilizzato per la stampa delle pagine visualizzate. È stato posto particolare accorgimento a questo tipo di funzione, omettendo le parti del sito meno importanti. È stato altresì preso in considerazione la rimozione dei contenuti quali sfondi decorativi per ottenere una stampa più pulita possibile. Invece abbiamo ritenuto buona cosa mantenere le immagini in quanto crediamo siano importanti per il nostro sito.
	
\end{itemize}

Il font utilizzano è di tipo standard e tutti i caratteri utilizzati hanno una dimensione relativa permettendoci una maggiore adattabilità alle preferenze dell'utente, senza influire negativamente nell'aspetto del sito.
\newpage
\section{Comportamento}
\newpage
\section{Gestione dati}
Tutti i dati sono immagazzinati nei file xml.
\begin{description}
	\item[Login.xml]\hfill \\
	È l'elenco delle combinazioni "username" + "password" che possono accedere alla zona di amministrazione del sito.
	\item[Prodotti.xml] \hfill\\È l'elenco dei prodotti disponibili e sono divisi in 3 categorie prefissate. Sia le categorie che i prodotti possiedono vari nodi che ne descrivono le caratteristiche. In particolare i prodotti possono essere disponibili sotto nessuna o più lavorazioni.
	\item[Lavorazioni.xml]\hfill \\ È l'elenco delle lavorazioni disponibili. Anch'esse possiedono vari nodi che ne descrivono le caratteristiche.
\end{description}
Per una gestione dati più semplice rispetto al modificare direttamente i file xml, il sito rende disponibile una serie di funzionalità nell'area di amministrazione, così che anche un utente inesperto possa effettuare aggiornamenti.
\newpage
\section{Perl}
Il linguaggio Perl è stato utilizzato sia per stampare modularmente e dinamicamente le varie pagine, sia per il corretto funzionamento dei servizi offerti dal sito, cioè il login ed i vari tipi di modifiche ai prodotti ed alle lavorazioni.
\\Le librerie utilizzate sono state:
\begin{itemize}
	\item  CGI
	\item Lib::XML
	\item Lib::XSLT
\end{itemize}
Riguardo alla stampa delle pagine, è stata sfruttata semplicemente la print assieme ad alcune pagine xhtml cosÏ da produrre l'opportuno output.
\\Riguardo al login, tramite l'apposito form il sito controlla in login.xml se l'username e la password corrispondono, restituendo un errore oppure permettendo l'accesso alla zona di amministrazione. Solo i gestori del sito hanno accesso a quest'area e non vi è modo di registrarsi.
\\Riguardo alla gestione dei prodotti e delle lavorazioni, sono offerte 5 funzionalità per chi non è pratico di xml che garantiscono il rispetto del corrispondente XMLSchema.
\\
\\Queste funzionalità si possono suddividere in due categorie dal codice molto simile:

\begin{description}
	\item[Inserimento] alla quale appartengono l'inserimento prodotto/lavorazione. Questa categoria crea un nodo con i parametri impostati dall'utente, lo controlla e lo inserisce nel rispettivo file xml.
	\item[Modifica] alla quale appartengono la modifica prodotto/lavorazione e la riassegnazione di una lavorazione ai diversi prodotti.
	\\Questa categoria si occupa di modificare od eliminare i nodi in base alle scelte dell'utente.  Inizialmente occorre selezionare l'elemento, poi verr‡ mostrata la pagina con i vari parametri modificabili. I cambiamenti vengono controllati ed introdotti nei rispettivi file xml.
	
\end{description}
\newpage
\section{Validazione}
\end{document}
