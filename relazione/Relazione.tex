\documentclass[11pt]{article}
\usepackage[utf8]{inputenc}
\usepackage{graphicx}
\usepackage[italian]{babel}
\usepackage{hyperref}
\title{Progetto di Tecnologie Web}


\begin{document}
	\maketitle
	\begin{figure}[h]
		\centering
		\includegraphics[width=0.7\linewidth]{logo-unipd.png}
	\end{figure}
	\
	\begin{center}{\fontsize{20}{10}\selectfont Autori:}\end{center}
	\begin{center}{\fontsize{20}{30}\selectfont 
			Marco Casagrande
			\\ Walter Sandon 1009138
			
			}\end{center}
	
	\newpage
	\tableofcontents
	\newpage
	\listoffigures
	\newpage
	
	\section{Abstract}
	
Il progetto sviluppato implementa il sito internet di una azienda che produce per conto terzi griglie metalliche ad uso domestico, quali griglie per forni e frigoriferi.
Il sito si propone di dare ogni informazione che possa essere rilevante agli utenti visitatori come i prodotti che la IMAS riesce a realizzare e le relative lavorazioni disponibili, la posizione della ditta  e la possibilità, nel caso si abbia la necessità di ulteriori delucidazioni, di potere contattere gli opratori tramite apposito form nell'area dedicata.
Inoltre gli amministratori hanno la possibilità di accedere ad un area riservata dalla quale modificare  o aggiungere i prodotti e lavorazioni in catalogo; in particolare lato prodotti:
\begin{itemize}
	\item inserire nuovo prodotto,
	\item assegnare a più prodotti una lavorazione, 
	\item modificare un prodotto esistente, in particolare:	
	\begin{itemize}
		\item nome
		\item foto
		\item attributo alt della foto
		\item lavorazione
		\item descrizione
	\end{itemize}
	\item eliminare un prodotto.
\end{itemize}
La lavorazione è similare al prodotto in quanto di possono eseguire le stesse operazioni di inserimento e modifica dei propri campi

\subsection{Combinazione dei colori}
Per cercare di garantire una totale usabilità del sito anche ad utenti affetti da problemi visivi è stato scelto di utilizzare uno schema a colori che esaltasse il contrasto tra sfondo e testo. Per testare tale scelta è stato utilizzato il servizio offerto da \href{http://colorfilter.wickline.org/}{\textit{wickline}} che mostra come il sito può venire visualizzato da utenti con determinati problemi. Di seguito vengono riportati i risultati ottenuti.
\newpage
\section{Utenti destinatari}
\paragraph{Privati/Aziende}
Una prima categoria di utenti destinatari del sito comprende in piccola misura i privati cittadini e nella maggioranza alle aziende che producono oggetti che rientrano nelle categorie merciologiche presenti nella sezione  \textit{Prodotti}.\\
Questi utenti potranno visualizzare il catalogo e constatare la presenza di un particolare prodotto; per questo, oltre alle informazioni di base del prodotto, si può trovare l'indicazione circa le possibili \textit{lavorazioni}.\\
Oltre a ciò, gli utenti potranno (nelle due pagine \textit{Home} e \textit{Contatti}) scoprire la storia della compagnia e qual'ora fossero interessati, chiedere informazioni tramite un apposito form.

\paragraph{Amministratori}
Un'altra categoria di utenti del sito è rappresentata dagli amministratori IMAS che potranno accedere, autenticandosi attraverso un link nell' \textit{header} a destra del logo aziendale, ad un'area riservata dalla quale è possibile gestire il catalogo dei prodotti e lavorazioni che può offrire la ditta.

\newpage

\section{Usabilità}

Dopo aver discusso col proponente riguardo alle possibili implementazioni del sito, in fase di progettazione ci siamo chiesti come distribuire le varie informazioni nel sito cercando di dare una giusta collocazione alle cose ponendo l'attenzione su due variabli quali le azioni che un utente maggiormente esegue e la tipologia aziendale in questione

\begin{description}
	\item [Consultare il catalogo prodotti] \hfill \\
	L'operazione più frequentemente compiuta dagli utenti è quella di consultare il catalogo dei prodotti presenti.
	\hfill 
	Essendo però una ditta che non ha una continuità di rinnovo dei prodotti, si è ritenuto apportuno non metterlo in evidenza nella homepage, ma piuttosto  dedicare un area apposita. 
	La \textbf{Home} è stata adibita a chi entra per la prima volta nel sito o piuttosto quache vecchi utente che vogliano informarsi circa sulle origini, sulla storia o sulla mission aziendale. \hfill
	Nel contempo in queste sezioni sono stati inseriti link diretti per poter consultare i nostri prodotti  ed è stata inoltre predisposta una comoda \textit{NavBar} con la quale raggiungere da ogni punto del sito le pagine \textbf{Prodotti} e \textbf{Lavorazioni}, i cui nomi sono esplicativi circa le informazioni contenute.
	Stessa considerazione va fatta per la \textbf{consultazione delle lavorazioni}.
	\item [Manutenzione del catalogo] \hfill \\
	L'operazione di manutenzione dei cataloghi (aggiunta, modifica e rimozione dei prodotti/lavorazioni) è riservata agli amministratori del sito.\\
	 È logico pensare che, essendo tali utenti un numero inferiore rispetto ai visitatori, il link per l'autenticazione e l'accesso all'area riservata possa essere messa in posizione \textit{nascosta}. Portroppo per ragioni di spazio l'abbiamo posizionata solo in alto a destra sapendo di creare così disorientamento.\\
	 Abbiamo cercato di limitare i danni avvisando,  una volta entrati nell'area login, che tale area è riservata ai dependenti IMAS.\\
	  \textit{Si noti che il pulsante per poter entrare nell'area gestione prodotti e lavorazioni è visibile solo quando  l'amministratore ha eseguito il login.}
\end{description}
\subsection{Elementi dell'Interfaccia Grafica}

Analizziamo ora gli elementi dell'interfaccia utilizzati per renderla il più chiara (intuitiva per ogni utente) e diretta (minor numero di click per un operazione) possibile.
\begin{description}
	\item [NavBar] La barra di navigazione è presente in ogni pagina e consente l'accesso diretto alla homepage, l'accesso alle pagine \textit{Prodotti}, \textit{Lavorazioni} e \textit{Contatti}.
	\item [Breadcrumbs] Per evitare che l'utente si senta disorientato all'interno del sito, quando dalla homepage si addentra con maggiore profondità nella struttura del sito, compare il percorso dell'utente a partire dalla homepage. Per approfondire la struttura si veda la sezione \hyperref[sub:Responsivity]{\underline{Responsive Design}}.
	\item [Link] I link sono tutti sottolineati e segnalati in giallo, quelli visitati invece in arancione.
\end{description}

\newpage
\section{Gerarchia file}
\newpage
\section{Architettura}
\subsection{Progettazione layout}
\subsection{Sviluppo layout}
\subsection{Layout per dispositivi mobili}
\subsection{Layout di stampa}

\newpage
\section{Struttura}
\newpage

\section{Presentazione}
\newpage
\section{Comportamento}
\newpage
\section{Gestione dati}
\newpage
\section{Perl}
\newpage
\section{Validazione}
\end{document}
